\section{Conditional Memory}

\frame{\frametitle{Reasoning–Retrieval Decomposition in Language Models}
 \begin{itemize}
   \item \lit{\citet{cheng_et_2026_cm}} formalize language modeling as comprising two qualitatively distinct sub-tasks: \alert{compositional reasoning} and \alert{knowledge retrieval}
   \item \lit{\citepos{vaswani_et_2017}} Transformer lacks an explicit retrieval mechanism; both reasoning and retrieval are implemented via matrix computation
   \item Attention-based computation scales as \alert{$\mathcal{O}(T^2 d)$}, where $T$ denotes sequence length and $d$ model dimension, whereas classical $n$-gram lookup achieves \alert{$\mathcal{O}(1)$} access
   \item \lit{\citet{cheng_et_2026_cm}} introduce the \alert{Engram module} to decouple retrieval from computation, enabling a more efficient allocation of computational budget across the two sub-tasks
 \end{itemize}
}


\frame{\frametitle{Engram Module $1/2$: Retrieving}
     \begin{itemize}
         \item $\mathcal{P}$ maps tokens to canonical identifiers, e.g., normalization
                \begin{align*}
                  x_t' = \mathcal{P}(x_t),\; t = 1,2,\ldots,T
                \end{align*}
         \item Construct suffix $N$-grams as local context descriptors
               \begin{align*}
                 g_{t,n} = \left(x_{t-n+1}',\ldots,x_t'\right),\;n = 2,\ldots,N
               \end{align*}
         \item Multi-head hashing approximates a large $N$-gram table without explicit enumeration
               \begin{align*}
                 z_{t,n,k} = \varphi_{n,k}(g_{t,n}),\; k = 1,\ldots, K
               \end{align*}
         \item Retrieve and concatenate learned memory embeddings
               \begin{align*}
                 e_{t,n,k}&=E_{n,k}(z_{t,n,k}),\;\forall n,k\\
                 \mathbf{e}_t &= \begin{bmatrix}
                                    e_{t,2,1} & \cdots & e_{t,2,K} &\cdots& & e_{t,N,1} & \cdots & e_{t,N,K}
                                 \end{bmatrix}
               \end{align*}
         \item Note, the above retrieval via multi-head hashing requires \alert{$\mathcal{O}(1)$}
      \end{itemize}
}


\frame{\frametitle{Engram Module $2/2$: Gating}
 \begin{itemize}
   \item Let $\mathbf{h}_t$ denote the hidden state from preceding attention layers, the gate measures semantic alignment between current global context and retrieved memory
       \begin{align*}
         &\alpha_t = \sigma\left(\frac{\text{RMSNorm}(\mathbf{h}_t)^\top \text{RMSNorm}(\mathbf{k}_t)}{\sqrt{d}}\right),\; \mathbf{k}_t = W_K\mathbf{e}_t
       \end{align*}
   \item If memory is irrelevant, $\alpha_t \to 0$. The context-aware gate filters the retrieved memory
       \begin{align*}
           \tilde{\mathbf{v}}_t = \alpha_t \mathbf{v}_t,\;\mathbf{v}_t = W_V\mathbf{e}_t
       \end{align*}
   \item A short convolution expands expressivity and injects local interaction
         \begin{align*}
           \mathbf{Y} = \text{SiLU}(\text{Conv1D}(\text{RMSNorm}(\tilde{\mathbf{V}})))+\tilde{\mathbf{V}}
         \end{align*}
   \item Residual structure preserves stability as usual
 \end{itemize}
}

\frame{\frametitle{Engram-Augmented Transformer}
 \begin{itemize}
   \item Let $\mathbf{X}$ be an input sequence, recall the computation graph of the standard Transformer
         \begin{align*}
           \mathbf{B} &= \text{LN}(\mathbf{X}+\text{Attention}(\mathbf{X}))\\
           \mathbf{H} &= \text{LN}(\mathbf{B}+\text{FFN}(\mathbf{B}))
         \end{align*}
   \item Given $\mathbf{H}$, the Engram module takes as well $\mathbf{X}$ to compute $\mathbf{Y}$, and then augments $\mathbf{H}$ through residual connection
         \begin{align*}
           \mathbf{Y} & = \text{Engram}(\mathbf{H},\mathbf{X}) \\
           \mathbf{H} &= \mathbf{H} + \mathbf{Y}
         \end{align*}
   \item Followed by the standard attention and MoE. Note that, if the gate suppresses memory, the block reduces to the standard Transformer
   \item The model can now allocate capacity between computation and retrieval instead of simulating both with depth alone
 \end{itemize}
} 