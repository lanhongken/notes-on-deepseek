%=================================================================================================================================================
% Template for Lecture Slides
%=================================================================================================================================================

\usetheme{Pittsburgh}
\usefonttheme{structuresmallcapsserif}
\definecolor{hublue}{cmyk}{1,.6,0,.2}
\definecolor{hured}{cmyk}{0,.9,.8,.4}
\definecolor{orchid}{rgb}{0.73, 0.33, 0.83}
\usecolortheme[cmyk={1,.6,0,.2}]{structure}
%\setbeamersize{text margin left=0.5cm, text margin right=0.5cm}
\setbeamercolor{alerted text}{fg=hured}
\newcommand{\lit}[1]{\textcolor{hublue}{#1}} % color literature with hublue
\setbeamertemplate{navigation symbols}{}
\setbeamerfont{frametitle}{size=\large}
\setbeamerfont{title}{size=\Large}
\setbeamerfont{date}{size=\small}
\setbeamertemplate{itemize items}[default]
\setbeamertemplate{enumerate items}[default]
\setbeamertemplate{caption}[numbered]
\setbeamertemplate{sections/subsections in toc}[sections numbered]
\setbeamertemplate{subsection in toc}[subsections numbered]
\setbeamertemplate{subsection in toc}{\vspace{1.5ex}\leavevmode\leftskip=3.5em\rlap{\hskip-2em\inserttocsectionnumber.\inserttocsubsectionnumber}\inserttocsubsection\vspace{0.5ex}\par}
\setbeamercolor{subsection in toc}{fg=hublue}
\setbeamerfont{subsection in toc}{family=\rmfamily}


%\AtBeginSection{\frame{\sectionpage}}
\AtBeginSection{\frame{\tableofcontents[currentsection]}}
\AtBeginSubsection{\frame{\tableofcontents[currentsection,currentsubsection]}}
\usepackage{graphicx}
\usepackage{amsmath}
\usepackage{amsthm}
\usepackage{thmtools}
\usepackage{amssymb}
\usepackage{ulem}
\usepackage{rotating}
\usepackage{verbatim}
\usepackage{comment}
\usepackage{hyperref}
\usepackage{multirow}
\usepackage{multicol}
\usepackage{beamerfoils}
\usepackage{booktabs}
\usepackage{mathrsfs}
\usepackage{pdfpages}
\usepackage{movie15}
\usepackage{arydshln} % for dashed lines in arrays
\usepackage{mathtools}% for norm
%\usepackage[framed,numbered,autolinebreaks,useliterate]{mcode} % for matlab code
\usepackage{bbm} % for indicator function etc.
\usepackage{bbold}
\usepackage{empheq} % box equations
\usepackage{tikz}% circle a number
\newcommand*\circled[1]{\tikz[baseline=(char.base)]{
            \node[shape=circle,draw,inner sep=2pt] (char) {#1};}}

\usepackage{algorithm}
\usepackage{algorithmic}
\usepackage{fancybox}

\usepackage[most]{tcolorbox}

\setbeamertemplate{blocks}[rounded][shadow=true] % use rounded blocks with standard beamer shadow



\setbeamertemplate{theorems}[numbered]
\theoremstyle{definition}
\newtheorem{exmp}{Example}[section]
\newtheorem{excursion}{Excursion}[section]

\DeclarePairedDelimiterX{\norm}[1]{\lVert}{\rVert}{#1} % norm
\DeclareMathOperator{\var}{var} % variance
\DeclareMathOperator{\std}{std} % variance
\DeclareMathOperator{\cov}{cov} % covariance
\DeclareMathOperator{\eig}{eig} % eigenvalue
\DeclareMathOperator{\vecop}{vec} % eigenvalue
\DeclareMathOperator{\tr}{tr} % eigenvalue
\DeclareMathOperator{\diag}{diag} % eigenvalue
\DeclareMathOperator{\vech}{vech} % eigenvalue
\DeclareMathOperator{\rank}{rank} % eigenvalue
\DeclareMathOperator{\prob}{prob} % eigenvalue
\DeclareMathOperator{\sign}{sign} % eigenvalue

\DeclareMathOperator*{\argmax}{arg\,max}
\DeclareMathOperator*{\argmin}{arg\,min}

\makeatletter
\newcommand{\distas}[1]{\mathbin{\overset{#1}{\kern\z@\sim}}}%
\newsavebox{\mybox}\newsavebox{\mysim}
\newcommand{\distras}[1]{%
  \savebox{\mybox}{\hbox{\kern3pt$\scriptstyle#1$\kern3pt}}%
  \savebox{\mysim}{\hbox{$\sim$}}%
  \mathbin{\overset{#1}{\kern\z@\resizebox{\wd\mybox}{\ht\mysim}{$\sim$}}}%
}
%\newcommand\harvardand{\&}

% References:
\usepackage[authoryear,semicolon]{natbib}
\makeatletter
\DeclareRobustCommand\citepos
  {\begingroup\def\NAT@nmfmt##1{{\NAT@up##1's}}%
   \NAT@swafalse\let\NAT@ctype\z@\NAT@partrue
   \@ifstar{\NAT@fulltrue\NAT@citetp}{\NAT@fullfalse\NAT@citetp}}
\makeatother
